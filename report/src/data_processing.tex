\section{Data pipeline}
\todo{Missing meta section}
\subsubsection{What features to focus on}
We chose to only be concerned with the "content" column for the base from which we want to predict the type. Whilst it
might be tempting to also include features such as "domain" as input, we found that this leads to massive over-fitting,
primarily because the dataset is constructed by labeling articles as true or false based on their domain (i.e. all
articles from the Washington Post being labeled true, versus all from Fox News being labeled as false) (**REF**).
\subsubsection{Grouping}
Whilst it is possible to use logistic regression for multi-class classification, we are in this case only interested in
a binary scenario of "fake news" or "reliable news". We therefore decided to encode the "type" of the article into 1's
and 0's, such that \textit{reliable} is encoded as 1 and the rest is encoded as 0. We've chosen this rather
conservative approach to what counts as reliable (ex. excluding "political" articles from being reliable), because it
would be of greater ethical concern if we labelled a fake article as true by mistake versus labeling a true article
fake by mistake.

\todo{Tokenization section, remember unigrams vs bigrams}

\subsubsection{TF-IDF feature extraction}
In order to extract useful features from the content column, we decided to run it through a TF-IDF (Term
Frequency - Inverse Document Frequency) model in order to identify the importance of different words for predicting
wether an article was reliable or not. We can then use the weights gained through the TF-IDF model as our input vector
in our logistic regression model. \todo{Maybe add the math behind TF-IDF and / or potentially the tokenization we did}

\subsubsection{Dimensionality reduction through Truncated Singular Value Decomposition}
After running the content through the TF-IDF model, we were left with a rather large input vector with a dimensionality
of 2048***
\todo{Correct?}. In order to consolidate similar features together, and to improve the performance we reduced the
dimensionality of our input using truncated singular value decomposition (REF) \todo{Remember to add reference}.
