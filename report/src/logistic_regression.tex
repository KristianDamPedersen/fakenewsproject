\section{Simple model: Logistic Regression (1/2 page)} 
For our baseline, we chose to use a logistic regression model. Much of the heavy lifting in developing this model took place in the previous sections,
so once we reached this step we chose to just use a relatively naive implementation, using the default arguments provided
by \textit{scikit-learn}. This meant we had the following hyperparameters for our model. \textbf{Penalty}: l2. This form of penalty adds a penalty that is equal to the square of the magnitude of the
    coefficients of the model. In other words, it tries to limit the influence that any single coefficient has on the
<<<<<<< HEAD
    model. \textbf{Solver}: lbfgs (limited memory BFGS). This solver is one of the most general logistic regression
    solvers, and is therefore a natural fit when we don't wish to direct it further.
=======
    model. \textbf{Solver}: lbfgs (limited memory BFGS). This solver is especially useful on "smaller" datasets (such as
    ours), both because it is quick to train, but also because it is memory efficient which helps us on limited
    hardware. TK REMOVE PRIOR SENTENCE???
>>>>>>> refs/remotes/origin/final_touches
