\section{Results and evaluation}
In this section we will discuss the performance of both our simple and complex models, and discuss how these stack up
against eachother. We will discuss the accuracy and performance on both the FakeNewsCorpus dataset and the LIAR
dataset. And we will end of this evaluation by discussing the challenges in general fake news classifiers, and the
ethical and practical challenges such a model encounters.

\subsection{Relative performance on FakeNewsCorpus}
Let's begin by looking at our models performance on the FakeNewsCorpus dataset. As we see from table \textbf{TABLE REF},
both of our models performed admirably when being run against the test set, with our more complex DNN pulling ahead by
about \textbf{x \%}. This can largely be contributed to the extra nuances that the complex model was able to detect in
the dataset.
\newline
\textbf{TABLE HERE}
\newline

\subsection{Generalisability and why the models are shit at LIAR}
\todo{This is where we include Josh' magnificent visualizations!}
However, whilst these results are impressive, they also hint that our models might be overfitted to our domain. This is
likely due to the nature of the dataset we used for training not being general enough, which means a broader selection
of data sources and data sets would be needed to achieve a more generalisable model. This issue is highlighted in
particular by the relatively terrible accuracy acheived on the LIAR dataset as seen in table \textbf{TABLE REF}.
\newline

\textbf{TABLE HERE}
\newline
In addition to challenges regarding the available data, there are other challenges that any fake news classifier
encounters as we will discuss next.

\subsection{Ethical considerations with fake news classification}
The subpar performance on the LIAR dataset highlights the difficulty in creating a proper fake news classifier. Many
companies have tried and failed to automatically moderate fake news due to the nuances that gets introduced when moving
from extremely domain specific models, to more generalised models working on multiple domains. When trying to
generalise a model, questions about what consitutes reliable, fake or misleading news quickly arises. Are all
politically motivated publications fake / unreliable? The IMF, European Central Bank, Red Cross and many more, are all
inherently political organisations, which we all still generally percieve as reliable. On the same token, publications
on eugenics from nazi era germany are generally seens as unreliable. Furthermore, does a publication raising awareness
of the real side-effects of the HPV vaccine count as reliable? And how does an algorithm distinguish this from a
publication spreading fake statistics of the corona vaccine?. Whilst there are publishers that we all (generally)
accept as being reliable, and others that are generally percieved as being unreliable, the exact line delineating the
two is hard (if not impossible) to strongly define.
