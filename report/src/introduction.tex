\section{Introduction}
Fake news has increasingly entered the public discourse in recent years. The Oxford English Dictionary coined the term
in 2013 (REF), and since then the use of the term has only gotten more and more common. We've encountered
novel challenges with regards to the spreading of false information, both through informal means (such as social media)
and increasingly through traditional media as well. Therefore, fake news classification has become a hot topic within AI,
but despite this the subject has proven difficult to tackle succesfully. In this technical report, we'll throw our hat
into the ring, by building a fake news classifier using 3 different complex models (a large deep neural network, a
smaller DNN and a model based on XGBoost). We'll start off by going over our
data cleaning and processing. We will then use logistic regression to create a simple model that we can use as a
baseline. We will then go over our choice of complex models, before we take a look at the results and compare
performance across the different models. We will also evaluate our models on a novel dataset (LIAR), to assess wether
they generalize to unfamiliar domains.
