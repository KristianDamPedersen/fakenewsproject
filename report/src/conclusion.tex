\section{Conclusion}
In this technical report we described the development of a Deep Neural Network for fake news classification. We started
by outlining the data import, exploration and processing that we did on the FakeNewsCorpus dataset, which included both
tokenization, feature extraction using TF-IDF and dimensionality reduction using Single Value Decomposition.. We then build a
simple baseline model using logistic regression to serve as our baseline which achieved a \textbf{X\%} on the dataset.
We then constructed a DNN that achieved a higher accuracy, reaching \textbf{X\%}. To test the generalisability of our
model, we applied them to a different fake news dataset (the LIAR dataset), and found that they both achieved a very
poor performance, only reaching \textbf{X\%} and \textbf{X\%} accuracy respectively. We conclude that we have managed
to build two efficient models that function well on the narrow (and potentially inaccurate) domain defined in the
FakeNewsCorpus dataset, but that generalises poorly to other domains such as that found in LIAR. Therefore, our
approach has failed to build a general fake news classifier, using the methods outlined above.
