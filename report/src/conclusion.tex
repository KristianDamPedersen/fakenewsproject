\section{Conclusion}

\subsection{Dataset issues}
As has been highlighted throughout this report, the FakeNewsCorpus dataset has many shortcomings, chief among which is
the classification of reliability based on domain. As we see, our classifiers trained on this dataset, generalise quite
poorly outside of their original domain.

\subsection{TF-IDF limitations}
Another key limitation is the use of TF-IDF model for embedding the content. The limitation is that TF-IDF does
not provide us with local context. It instead provides us with a single embedding representing the important keywords of the
entire document, without understanding the semantics of individual sentences. Here, a more complex word embedding model
would be a sensible alternative to try.

\subsection{Ethical considerations with fake news classification}
The subpar performance on the LIAR dataset highlights the difficulty in creating a proper fake news classifier. When trying to
generalise a model, questions about what consitutes reliable, fake or misleading news quickly arises. Does a publication raising awareness
of the real side-effects of the HPV vaccine count as reliable? And how does an algorithm distinguish this from a
publication spreading fake statistics of the corona vaccine? Whilst there are publishers that we all (generally)
accept as being reliable, and others that are generally percieved as being unreliable, the exact line delineating the
two is hard (if not impossible) to strongly define.


In this technical report we described the exploration of three different models for fake news classification. We started
by outlining the data import, exploration and processing we applied to the FakeNewsCorpus dataset. The processing
included both tokenisation, feature extracting using TF-IDF and dimensionality reduction through single value
decomposition. We build a logistic regression model to serve as our baseline, and then constructed three more complex
models (two DNNs and an XGBoost model). Whilst all of the models performed admirably on the FakeNewsCorpus dataset, they
all failed to generalise when we tested them against another dataset (LIAR). As discussed in our evaluation section,
this is likely due both to the quality of our dataset and the inherent difficulties in constructing fake news
classifiers in general. In the end we conclude that our approach has failed to build a general fake news classifier, and
that more sophisticated tools need to be incorporated in order to achieve generalisable success.

